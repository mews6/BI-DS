% Auth: Nicklas Vraa
% Docs: https://github.com/NicklasVraa/LiX

\documentclass{textbook}

\lang      {english}
\title     {Business Intelligence and Analytics}
\subtitle  {(For the Layman)}
\author    {Jaime Torres}
\cover*    {resources/textbook_front.pdf}{resources/textbook_back.pdf}
\license   {CC}{by-nc-sa}{3.0}{}
\publisher {Nobody, just me!}
\edition   {1}{2024}
\dedicate  {You guys}{Because of your will to learn}
\thank     {<3}
\keywords  {optimization, simulation, python, mathematical modeling}

\begin{document}

\tableofcontents

\part{Machine Learning.}

\chapter{Introduction to Machine Learning}

In order to work on machine learning, we must understand the nature of the field. As it comes to be understood, a 
first assumption we'll see is to assume 'Artificial Intelligence' as a sort of synonym. However, there is a distinction
to be made for the sake of scientific accuracy, as the hope Artificial Intelligence has, as it is the emulation of human 
intelligence in problem solving might be quite broader to the scope of current public scientific knowledge. 

a definition of Machine Learning, for our purpouses is:

"The science of programming computers in order for them to learn according to their data"

This subset of the field of Artificial Intelligence, will by it's own, become a whole area of study with it's own subsets, such as 
deep learning. We'll talk about such systems further in a while, however for now, we must focus on the usefulness of this discipline, as 
it happens to be amazingly useful for the sake of understanding, manipulating and managing data in 

\section{Introduction to the subsections of Machine Learning}

Without going much into detail on the techniques that Machine Learning tends to implement, we can see their 
classification on the 

\subsection{Supervised Learning}

Supervised learning is a technique where we have labels that classify our data for further 
analysis of data that is not in the dataset.

\chapter{The ASUM-DM methodology}

For applications in business intelligence of deep learning, we'll generally apply
ASUM-DM. Developed by IBM in an effort to improve their previous 'CRoss
Industry Standard Process for Data Mining (CRISP-DM)'.   


\part{Data Warehouses.}

\end{document}
